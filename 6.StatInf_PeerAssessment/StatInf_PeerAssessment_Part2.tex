\documentclass[]{article}
\usepackage{lmodern}
\usepackage{amssymb,amsmath}
\usepackage{ifxetex,ifluatex}
\usepackage{fixltx2e} % provides \textsubscript
\ifnum 0\ifxetex 1\fi\ifluatex 1\fi=0 % if pdftex
  \usepackage[T1]{fontenc}
  \usepackage[utf8]{inputenc}
\else % if luatex or xelatex
  \ifxetex
    \usepackage{mathspec}
  \else
    \usepackage{fontspec}
  \fi
  \defaultfontfeatures{Ligatures=TeX,Scale=MatchLowercase}
\fi
% use upquote if available, for straight quotes in verbatim environments
\IfFileExists{upquote.sty}{\usepackage{upquote}}{}
% use microtype if available
\IfFileExists{microtype.sty}{%
\usepackage{microtype}
\UseMicrotypeSet[protrusion]{basicmath} % disable protrusion for tt fonts
}{}
\usepackage[margin=1in]{geometry}
\usepackage{hyperref}
\hypersetup{unicode=true,
            pdftitle={Inferential Data Analysis of Tooth Growth},
            pdfauthor={Omar Nooreddin},
            pdfborder={0 0 0},
            breaklinks=true}
\urlstyle{same}  % don't use monospace font for urls
\usepackage{color}
\usepackage{fancyvrb}
\newcommand{\VerbBar}{|}
\newcommand{\VERB}{\Verb[commandchars=\\\{\}]}
\DefineVerbatimEnvironment{Highlighting}{Verbatim}{commandchars=\\\{\}}
% Add ',fontsize=\small' for more characters per line
\usepackage{framed}
\definecolor{shadecolor}{RGB}{248,248,248}
\newenvironment{Shaded}{\begin{snugshade}}{\end{snugshade}}
\newcommand{\KeywordTok}[1]{\textcolor[rgb]{0.13,0.29,0.53}{\textbf{#1}}}
\newcommand{\DataTypeTok}[1]{\textcolor[rgb]{0.13,0.29,0.53}{#1}}
\newcommand{\DecValTok}[1]{\textcolor[rgb]{0.00,0.00,0.81}{#1}}
\newcommand{\BaseNTok}[1]{\textcolor[rgb]{0.00,0.00,0.81}{#1}}
\newcommand{\FloatTok}[1]{\textcolor[rgb]{0.00,0.00,0.81}{#1}}
\newcommand{\ConstantTok}[1]{\textcolor[rgb]{0.00,0.00,0.00}{#1}}
\newcommand{\CharTok}[1]{\textcolor[rgb]{0.31,0.60,0.02}{#1}}
\newcommand{\SpecialCharTok}[1]{\textcolor[rgb]{0.00,0.00,0.00}{#1}}
\newcommand{\StringTok}[1]{\textcolor[rgb]{0.31,0.60,0.02}{#1}}
\newcommand{\VerbatimStringTok}[1]{\textcolor[rgb]{0.31,0.60,0.02}{#1}}
\newcommand{\SpecialStringTok}[1]{\textcolor[rgb]{0.31,0.60,0.02}{#1}}
\newcommand{\ImportTok}[1]{#1}
\newcommand{\CommentTok}[1]{\textcolor[rgb]{0.56,0.35,0.01}{\textit{#1}}}
\newcommand{\DocumentationTok}[1]{\textcolor[rgb]{0.56,0.35,0.01}{\textbf{\textit{#1}}}}
\newcommand{\AnnotationTok}[1]{\textcolor[rgb]{0.56,0.35,0.01}{\textbf{\textit{#1}}}}
\newcommand{\CommentVarTok}[1]{\textcolor[rgb]{0.56,0.35,0.01}{\textbf{\textit{#1}}}}
\newcommand{\OtherTok}[1]{\textcolor[rgb]{0.56,0.35,0.01}{#1}}
\newcommand{\FunctionTok}[1]{\textcolor[rgb]{0.00,0.00,0.00}{#1}}
\newcommand{\VariableTok}[1]{\textcolor[rgb]{0.00,0.00,0.00}{#1}}
\newcommand{\ControlFlowTok}[1]{\textcolor[rgb]{0.13,0.29,0.53}{\textbf{#1}}}
\newcommand{\OperatorTok}[1]{\textcolor[rgb]{0.81,0.36,0.00}{\textbf{#1}}}
\newcommand{\BuiltInTok}[1]{#1}
\newcommand{\ExtensionTok}[1]{#1}
\newcommand{\PreprocessorTok}[1]{\textcolor[rgb]{0.56,0.35,0.01}{\textit{#1}}}
\newcommand{\AttributeTok}[1]{\textcolor[rgb]{0.77,0.63,0.00}{#1}}
\newcommand{\RegionMarkerTok}[1]{#1}
\newcommand{\InformationTok}[1]{\textcolor[rgb]{0.56,0.35,0.01}{\textbf{\textit{#1}}}}
\newcommand{\WarningTok}[1]{\textcolor[rgb]{0.56,0.35,0.01}{\textbf{\textit{#1}}}}
\newcommand{\AlertTok}[1]{\textcolor[rgb]{0.94,0.16,0.16}{#1}}
\newcommand{\ErrorTok}[1]{\textcolor[rgb]{0.64,0.00,0.00}{\textbf{#1}}}
\newcommand{\NormalTok}[1]{#1}
\usepackage{longtable,booktabs}
\usepackage{graphicx,grffile}
\makeatletter
\def\maxwidth{\ifdim\Gin@nat@width>\linewidth\linewidth\else\Gin@nat@width\fi}
\def\maxheight{\ifdim\Gin@nat@height>\textheight\textheight\else\Gin@nat@height\fi}
\makeatother
% Scale images if necessary, so that they will not overflow the page
% margins by default, and it is still possible to overwrite the defaults
% using explicit options in \includegraphics[width, height, ...]{}
\setkeys{Gin}{width=\maxwidth,height=\maxheight,keepaspectratio}
\IfFileExists{parskip.sty}{%
\usepackage{parskip}
}{% else
\setlength{\parindent}{0pt}
\setlength{\parskip}{6pt plus 2pt minus 1pt}
}
\setlength{\emergencystretch}{3em}  % prevent overfull lines
\providecommand{\tightlist}{%
  \setlength{\itemsep}{0pt}\setlength{\parskip}{0pt}}
\setcounter{secnumdepth}{0}
% Redefines (sub)paragraphs to behave more like sections
\ifx\paragraph\undefined\else
\let\oldparagraph\paragraph
\renewcommand{\paragraph}[1]{\oldparagraph{#1}\mbox{}}
\fi
\ifx\subparagraph\undefined\else
\let\oldsubparagraph\subparagraph
\renewcommand{\subparagraph}[1]{\oldsubparagraph{#1}\mbox{}}
\fi

%%% Use protect on footnotes to avoid problems with footnotes in titles
\let\rmarkdownfootnote\footnote%
\def\footnote{\protect\rmarkdownfootnote}

%%% Change title format to be more compact
\usepackage{titling}

% Create subtitle command for use in maketitle
\newcommand{\subtitle}[1]{
  \posttitle{
    \begin{center}\large#1\end{center}
    }
}

\setlength{\droptitle}{-2em}

  \title{Inferential Data Analysis of Tooth Growth}
    \pretitle{\vspace{\droptitle}\centering\huge}
  \posttitle{\par}
    \author{Omar Nooreddin}
    \preauthor{\centering\large\emph}
  \postauthor{\par}
      \predate{\centering\large\emph}
  \postdate{\par}
    \date{11/21/2018}


\begin{document}
\maketitle

knitr::opts\_chunk\$set(fig.height=4, fig.width=5) \#Synopsis This paper
will endeavour to perform basic inferential data analysis on the Tooth
Growth data set. More precisely it will assess the effects of vitamin C
on tooth growth in guinea pigs, and will try and conclude there's a
strong relationship or not.

\section{Data Set Summary}\label{data-set-summary}

The data set will this study will be using is the ToothGrowth data set.
The following is the description of the data given with dataset (in R):

The response is the length of odontoblasts (cells responsible for tooth
growth) in 60 guinea pigs. Each animal received one of three dose levels
of vitamin C (0.5, 1, and 2 mg/day) by one of two delivery methods,
orange juice or ascorbic acid (a form of vitamin C and coded as VC).

We can ascertain the above by looking at the dimension of the dataset:

\begin{Shaded}
\begin{Highlighting}[]
\KeywordTok{data}\NormalTok{(}\StringTok{"ToothGrowth"}\NormalTok{)}
\KeywordTok{dim}\NormalTok{(ToothGrowth)}
\end{Highlighting}
\end{Shaded}

\begin{verbatim}
## [1] 60  3
\end{verbatim}

Delving deeper into each variable of the data:

\begin{Shaded}
\begin{Highlighting}[]
\KeywordTok{unique}\NormalTok{(ToothGrowth}\OperatorTok{$}\NormalTok{supp)}
\end{Highlighting}
\end{Shaded}

\begin{verbatim}
## [1] VC OJ
## Levels: OJ VC
\end{verbatim}

\begin{Shaded}
\begin{Highlighting}[]
\KeywordTok{unique}\NormalTok{(ToothGrowth}\OperatorTok{$}\NormalTok{dose)}
\end{Highlighting}
\end{Shaded}

\begin{verbatim}
## [1] 0.5 1.0 2.0
\end{verbatim}

We can see from the above, that there are two supplements(delivery
methods): VC (Vitamin C) and OJ (Orange Juice). As for the dosages, it
is indeed as per description either: 0.5, 1.0 or 2.0 mg/day of Vitamin
C.

\section{Exploring the Data Set}\label{exploring-the-data-set}

To get a general idea of the effects of the different dosages on teeth
length (split by delivery method/supplement) we are going to conisder
the following plot:

\begin{Shaded}
\begin{Highlighting}[]
\KeywordTok{coplot}\NormalTok{(len }\OperatorTok{~}\StringTok{ }\NormalTok{dose }\OperatorTok{|}\StringTok{ }\NormalTok{supp, }\DataTypeTok{data =}\NormalTok{ ToothGrowth, }\DataTypeTok{panel =}\NormalTok{ panel.smooth, }\DataTypeTok{xlab =} \StringTok{"Length vs Dose, by supplement"}\NormalTok{)}
\end{Highlighting}
\end{Shaded}

\includegraphics{StatInf_PeerAssessment_Part2_files/figure-latex/unnamed-chunk-3-1.pdf}

It can be noted that the higher the dosage the more effect it has on
tooth growth. Furthermore we can see the different effects for each
delivery method on tooth growth, which can be verified by a box plot:

\begin{Shaded}
\begin{Highlighting}[]
\KeywordTok{boxplot}\NormalTok{(ToothGrowth}\OperatorTok{$}\NormalTok{len }\OperatorTok{~}\StringTok{ }\NormalTok{ToothGrowth}\OperatorTok{$}\NormalTok{supp, }\DataTypeTok{main=}\StringTok{"Boxplot of Length split by Supplement"}\NormalTok{)}
\end{Highlighting}
\end{Shaded}

\includegraphics{StatInf_PeerAssessment_Part2_files/figure-latex/unnamed-chunk-4-1.pdf}

It is evident that length of teeth that had OJ as a delivery method is
higher.

As such we will be investgating the following two hypotheses:

\begin{itemize}
\tightlist
\item
  Mean tooth growth by OJ is more than VC, while holding the dose
  constant
\item
  The higher the dose, the more it promotes tooth growth, while holding
  the delivery method constant
\end{itemize}

\section{Hypothesis}\label{hypothesis}

To ascertain our assumptions above, we will perform confidence test
using the \textbf{t-test}. Though for this we will check that this data
is normally distrubted, and we will ascertain this by a Q-Q plot:

\begin{Shaded}
\begin{Highlighting}[]
\KeywordTok{qqnorm}\NormalTok{(ToothGrowth}\OperatorTok{$}\NormalTok{len)}
\KeywordTok{qqline}\NormalTok{(ToothGrowth}\OperatorTok{$}\NormalTok{len, }\DataTypeTok{col=}\StringTok{"red"}\NormalTok{)}
\end{Highlighting}
\end{Shaded}

\includegraphics{StatInf_PeerAssessment_Part2_files/figure-latex/unnamed-chunk-5-1.pdf}

Following from above we will be testing whether method of delivery
(supp) has any effect (i.e OJ vs VC) while holding the dose constant at
0.5, 1.0 and 2.0. Our null hypothese \textbf{\(H_0:μ_{oj} = μ_{vc}\)},
while our alternative hypothesis will be
\textbf{\(H_a:μ_{oj} > μ_{vc}\)}. We are going to carry out a t-test and
set our significance level \textbf{α} to 0.05 (note all code is in the
Appendix):

\begin{longtable}[]{@{}ll@{}}
\toprule
dose constant at & p-value OJ v VC\tabularnewline
\midrule
\endhead
0.5 & 0.0031793\tabularnewline
1.0 & 0.0005192\tabularnewline
2.0 & 0.5180742\tabularnewline
\bottomrule
\end{longtable}

The above table informs that when holding the dose constant, for example
at 0.5, when performing a t-test between \(μ_{oj}\) and \(μ_{vc}\), it
will give us a p-value as reported in the table.

For the second hypothesis, we will be testing whether a higher dose will
have higher effect on mean toothgrowth, whilst holding the delivery
method (supp) constant. Therefore, our hyptheses are as follows, null
hypothesis: \textbf{\(H_0:μ_{0.5} = μ_{1.0}\)}, our alternative
hypothsis: \textbf{\(H_:μ_{0.5} < μ_{1.0}\)}. We will repeat the test
between dose 1.0 and 2.0 as well. As per previous test we are going to
set our sigfcance level \textbf{α} to 0.05:

\begin{longtable}[]{@{}lll@{}}
\toprule
supp method & p-value 0.5 v 1.0 & p-value 1.0 v 2.0\tabularnewline
\midrule
\endhead
OJ & 4.39245952758075e-05 & 0.0195975710231221\tabularnewline
VC & 3.40550885143253e-07 & 4.57780152831932e-05\tabularnewline
\bottomrule
\end{longtable}

\emph{For complete of both tests code please check Appendix}

\section{Conclusion}\label{conclusion}

Following our T Tests performed above, almost all of them have show a
p-value lower than the our signficance level \textbf{α} of 0.05. In
other words, most tests have rejected the null hypothesis.

We can conclude the following, that OJ as a delivery method had a
positive effect on tooth growth when compared to VC. The only exception
is when the dose was constant at 2.0 mg, where the p-value was quite
high, i.e confirming our null hypothesis.

Also for the dose, the higher the dose, the higher the tooth growth
confirming our alternative hypothese \(H_a\).

hello


\end{document}
